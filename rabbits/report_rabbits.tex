\documentclass[11pt]{article}

\usepackage{amsmath}
\usepackage{textcomp}
\usepackage[top=0.8in, bottom=0.8in, left=0.8in, right=0.8in]{geometry}
% Add other packages here %



% Put your group number and names in the author field %
\title{\bf Excercise 1.\\ Implementing a first Application in RePast: A Rabbits Grass Simulation.}
\author{Group \textnumero 17: Ogier Bouvier, Val\'erian Rousset}

\begin{document}
\maketitle

\section{Implementation}

\subsection{Assumptions}
% Describe the assumptions of your world model and implementation (e.g. is the grass amount bounded in each cell) %

\begin{itemize}
        \item When a rabbits reaches an amount of energy greater or
          equal to the birth threshold parameter, a new rabbit is
          created and placed randomly on the grid.
        \item Each rabbit dies when its energy reaches an amount lower
          than the death threshold parameter.
        \item Eating grass gives the rabbit an amount of energy
          corresponding to the amount of grass eaten.
        \item Additionally at each step the rabbits lose a certain
          amount of energy determined by the ``Aging rate'' parameter.
	\item At each step the rabbits decide if they are going to
          move this turn. If so they then choose a random direction
          out of the four possible. The rabbits eat the grass upon
          arriving on the new cell but also eats the grass on the cell
          it is on if it decides not to move.
        \item The grass can grow infinitely high on each cell and is
          only reset to 0 when a rabbit eats it.
\end{itemize}

\subsection{Implementation Remarks}
% Provide important details about your implementation, such as handling of boundary conditions %

\begin{itemize}
        \item  If no space is available when a rabbit gives birth,
          no rabbit is added but the energy is still consumed.

	\item color of each rabbit are for the level of energy
	\begin{itemize}
		\item yellow for enery between 10 and 30
		\item green for enery between 30 and 60
		\item blue else
	\end{itemize}
\end{itemize}

\section{Results}
% In this section, you study and describe how different variables (e.g. birth threshold, grass growth rate etc.) or combinations of variables influence the results. Different experiments with diffrent settings are described below with your observations and analysis

\subsection{Experiment 1}

\subsubsection{Setting}

\begin{itemize}
  \item Birth threshold 70
  \item Death threshold 20
  \item Growth rate 5
  \item Aging rate 4
  \item Initial rabbit count 100
\end{itemize}

\subsubsection{Observations}
% Elaborate on the observed results %
The number of rabbits quickly grows to the saturation point which
seems to be around 350 rabbits and then oscillates around this
number.

\subsection{Experiment 2}

\subsubsection{Setting}
\begin{itemize}
  \item Growth rate 3
  \item Aging rate 5
  \item Initial rabbit count 350
  \item World size 20x20
  \item Birth threshold 50
  \item Death threshold 0
\end{itemize}

\subsubsection{Observations}
% Elaborate on the observed results %
The number of rabbits falls down very quickly at the start due to a
shortage of grass. Once the population is small enough and there is
again enough grass it starts to grow back and stabilizes at around 200
rabbits.

\subsection{Experiment 3}

\subsubsection{Setting}

\subsubsection{Observations}
% Elaborate on the observed results %

\end{document}
