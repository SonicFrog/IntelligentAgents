\documentclass[11pt]{article}

\usepackage{amsmath}
\usepackage{textcomp}

% Add other packages here %


% Put your group number and names in the author field %
\title{\bf Excercise 3\\ Implementing a deliberative Agent}
\author{Group \textnumero 17 : Bouvier Ogier, Rousset Valérian}


% N.B.: The report should not be longer than 3 pages %


\begin{document}
\maketitle

\section{Model Description}

\subsection{Intermediate States}
The state is represented as follows:
\begin{itemize}
  \item The vehicle's current city
  \item The set of tasks currently carried by the vehicle
  \item The set of task currently waiting to be picked up by the agent
\end{itemize}

\subsection{Goal State}

% TODO: More details?

A goal state is basically like any other intermediate state except
its set of tasks carried and its set of tasks waiting to be picked up
are both empty implying there are no more tasks to pick up or deliver.

\subsection{Actions}

% TODO: Other possible actions?
The possible transitions between states are the following:

\begin{itemize}
    \item The vehicle moves to a target city to pickup a task there.
      The vehicle's current city is adjusted to reflect the move and the
      corresponding task is moved from the waiting to be picked up set to
      the carrying set.
    \item The vehicle moves to a target city to deliver a task there.
      As before the vehicle's current city is adjusted to reflect the
      move and the corresponding task is removed from the carrying
      set.
\end{itemize}

These transitions are repeated until there are no more tasks in either
the waiting to be picked up set or the carrying set at which point we
have reached a goal state.

When we move from city to city, we try to deliver any packet on the way.

\section{Implementation}

\subsection{BFS}
% Details of the BFS implementation %

% FIXME: Possible errors in implementation description?

The BFS implementation is quite simple. We start from a state where
the carrying set is empty, the current city is the vehicle's start
city and the waiting to be picked up set is made of all tasks known to
the agent. We then iterate by generating all the successors (children)
of our current state and exploring them one by one while checking
whether any of them are goal states. As soon as we reach a goal state
we stop and return its plan as our optimal plan.

\subsection{A*}
The A* implementation works similarly to the BFS implementation with a
few major differences. First instead of having only a set of states
we have a priority queue which contains the state we have yet to
visit. This way we visit the state which are most promising first and
avoid a lot of unnecessary computations.

\subsection{Heuristic Function}
% Details of the heuristic functions: main idea, optimality,
% admissibility %

% FIXME: Change heuristic to fix described drawback?

Our heuristic is based on the assumption that we have to travel at
most the maximum distance to a delivery location and/or at most the
maximum distance to a pickup location. It follows that a logical
heuristic to use is: maxDistanceToPickup + maxDistanceToDelivery.

The only issue with this heuristic is that it does not account for the
number of package yet to deliver, and will consistently underestimate
the distance to goal states. However A* is faster when the distance to
a goal state is underestimated.

\section{Results}

\subsection{Experiment 1: BFS and A* Comparison}
% Compare the two algorithms in terms of: optimality, efficiency, limitations %
% Report the number of tasks for which you can build a plan in less than one minute %

\subsubsection{Setting}
% Describe the settings of your experiment: topology, task configuration, etc. %

\subsubsection{Observations}
% Describe the experimental results and the conclusions you inferred from these results %


\subsection{Experiment 2: Multi-agent Experiments}
% Observations in multi-agent experiments %

\subsubsection{Setting}
% Describe the settings of your experiment: topology, task configuration, etc. %

\subsubsection{Observations}
% Describe the experimental results and the conclusions you inferred from these results %

\end{document}
