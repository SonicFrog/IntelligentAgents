\documentclass[11pt]{article}

\usepackage{amsmath}
\usepackage{textcomp}

% Add other packages here %


% Put your group number and names in the author field %
\title{\bf Excercise 4\\ Implementing a centralized agent}
\author{Group \textnumero 17 : Bouvier Ogier, Rousset Valérian}


% N.B.: The report should not be longer than 3 pages %


\begin{document}
\maketitle

\section{Solution Representation}

\subsection{Variables}
% Describe the variables used in your solution representation %

Our solution representation consist of a sequence of pickup and delivery
action for each vehicle in the simulation. Our program then handles
generating a complete move, pickup and delivery plan from these pickup
and delivery action for each vehicle. If the sequence of pickup and
delivery action sequence satisfies the constraints this always results
in a valid plan for the TaskSet.

\subsection{Constraints}
% Describe the constraints in your solution representation %
Our constraints are pretty simple. There are only two ways one
solution is invalid. One case is when a task is delivered by a vehicle
before this vehicle has picked the task up and the other is when a
vehicle carries more than its carrying weight at one point during its
plan.
So when applying a transformation to our current solution we check for
both of these condition and if one is not satisfied we reject the
resulting solution as invalid.


\subsection{Objective function}
% Describe the function that you optimize %
The objective function is a sum of the total cost of the plan for each
vehicle in the simulation. We compute the total distance of the plan
for each vehicle and we multiply it by the cost per kilometer of the
vehicle associated with the plan. This is a very precise heuristic
since it's the actual cost associated with the solution instead of an
approximation.

\section{Stochastic optimization}

\subsection{Initial solution}
% Describe how you generate the initial solution %
The generation of the initial solution is pretty simple. We just
assign every task to the vehicle with the biggest carrying
capacity until we have reached its maximum carrying capacity. We then
spill the tasks to the second vehicle with biggest carrying capacity
and so on. This generate very quickly an initial solution that has a
reasonable cost.

\subsection{Generating neighbours}
% Describe how you generate neighbors %

When generating neighbors we have two possible transformations from
our current solution. We can either swap two actions for the same
vehicle or swap a task from one vehicle to another. Applying either
one of this transformation at random (randomly selecting two vehicle
and randomly selecting a task to transfer between the two or randomly
selecting two action to swap for a random vehicle) for a random
vehicle gives a neighboring state for which we compute the cost and
retain it if the cost is lower than our current solution.


\subsection{Stochastic optimization algorithm}
% Describe your stochastic optimization algorithm %
Working from our initial solution we try to randomly apply the two
possible transformation to our current solution and retain the
resulting solution if it has a lower cost than our current
solution. We continue applying transformations and keeping the better
solution until we reach the limit of iteration or the limit of time
allocated for the algorithm.

\section{Results}

\subsection{Experiment 1: Model parameters}

\subsubsection{Setting}
% Describe the settings of your experiment: topology, task configuration, number of tasks, number of vehicles, etc. %
% and the parameters you are analyzing %
The only parameters which are not related to the topology, the number
of tasks and the number of vehicles of the simulation are the maximum
iteration count and the maximum amount of time for the plan
computation. We will now try to see if giving more time and more
iteration for our plan computation yields a better solution.

Number of tasks 30
Number of vehicles 4
Topology: England

\subsubsection{Observations}
% Describe the experimental results and the conclusions you inferred from these results %
We first try with 100'000 iterations:

The total plan cost for this number of iterations is ~27'000.

We then try with 1.000.000 iterations:

The total plan cost for this number of iterations is ~25'000.

We now try with 10.000.000 iterations:

The total plan cost for this number of iterations is ~26'000.

We can then conclude that 100'000 iterations seem to be sufficient
to find an optimal solution since multiplying the number of iteration
by 10 from this point yields an only slightly better solution!

\subsection{Experiment 2: Different configurations}
% Run simulations for different configurations of the environment (i.e. different tasks and number of vehicles) %

\subsubsection{Setting}
% Describe the settings of your experiment: topology, task configuration, number of tasks, number of vehicles, etc. %

For this experiment we used 100'000 iterations, a number of task of 20
and 5 vehicles.

In this configuration the first vehicle performs almost all the
tasks while two of the vehicles have no task assigned! The total cost
of the plan is ~18'000.

For the next experiment we used 100'000 iterations, a number of task of
40 and 4 vehicles. In that case the first vehicle performs the
majority of the task and the total cost plan is ~28'000.

\subsubsection{Observations}
% Describe the experimental results and the conclusions you inferred from these results %
% Reflect on the fairness of the optimal plans. Observe that optimality requires some vehicles to do more work than others. %
% How does the complexity of your algorithm depend on the number of vehicles and various sizes of the task set? %

We notice that in both cases there is one vehicle that performs a lot
more tasks than the others (almost more than the others
combined!). The complexity  of the algorithm is exponential in the
number of vehicles and tasks since every added task and every added
vehicle yield a whole new solution space to explore!

\end{document}
