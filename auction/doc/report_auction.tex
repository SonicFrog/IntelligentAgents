\documentclass[11pt]{article}

\usepackage{amsmath}
\usepackage{textcomp}
\usepackage[top=0.8in, bottom=0.8in, left=0.8in, right=0.8in]{geometry}
% add other packages here

% put your group number and names in the author field
\title{\bf Exercise 5: An Auctioning Agent for the Pickup and Delivery Problem}
\author{Group \textnumero 17: Ogier Bouvier, Valérian Rousset}

\begin{document}
\maketitle

\section{Bidding strategy}
% describe in details your bidding strategy. Also, focus on answering the following questions:
% - do you consider the probability distribution of the tasks in defining your strategy? How do you speculate about the future tasks that might be auctions?
% - how do you use the feedback from the previous auctions to derive information about the other competitors?
% - how do you combine all the information from the probability distribution of the tasks, the history and the planner to compute bids?
The basis for the bidding strategy is the same as used in a standard TCP stream
to find the windows size: additive increase, multiplicative decrease. The idea
is to have two signals, one meaning that we should decrease the bid (when we
lose one bet), the other that we should increase the bid (when we win one).

We use this found multiplier to increase or decrease the bid which is mainly the
cost.

Then, when all the task are allocated, we use a centralized algorithm to
dispatch it on both vehicles. That way we can bid more since our cost is
optimized and we make more reward.

\section{Results}
% in this section, you describe several results from the experiments with your auctioning agent

\subsection{Experiment 1: Comparisons with dummy agents}
% in this experiment you observe how the results depends on the number of tasks auctioned. You compare with some dummy agents and potentially several versions of your agent (with different internal parameter values). 

\subsubsection{Setting}
% you describe how you perform the experiment, the environment and description of the agents you compare with
We use two other agents, each dummy, with 100 tasks in total.

\subsubsection{Observations}
% you describe the experimental results and the conclusions you inferred from these results
The more task there is, the best our agent gets. With 20 tasks, we were overall
better, but there was some stabilisation time, however with 100 tasks, there was
clearly a better reward per kilometers from the beginning.

Because we have more tasks, that mean that we can win more bids and optimising
our allocations to each vehicle.

\subsection{Experiment 2: Comparisons with more dummy}
% other experiments you would like to present (for example, varying the internal parameter values)

\subsubsection{Setting}

We remove one agent from our company and have the others being dummy. Thus we
have three company, two with two dummy agents, one with a single auction agent.

\subsubsection{Observations}

We still have a better reward per kilometers than the dummy companies.

This means our bidding strategy allows us to get more than tasks than the dummy
agents, and we even make more profit than them.

\end{document}
